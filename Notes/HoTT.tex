\documentclass[11pt]{article}

\usepackage{amsthm, amssymb}
\theoremstyle{definition}
\newtheorem{definition}{Definition}[section]
\newtheorem{theorem}{Theorem}[section]

% Technical.
\renewcommand{\phi}{\varphi}
\newcommand{\txt}[1]{\texttt{#1}}
\newcommand{\text}[1]{\texttt{#1}}
\renewcommand{\(}{\left(}
\renewcommand{\)}{\right)}

% Definitions.
\newcommand{\defn}{:\equiv}
\newcommand{\defp}{:=}

% Universes.
\newcommand{\U}{\mathcal{U}}

\newcommand{\isContr}{\text{isContr}}
\newcommand{\isProp}{\text{isProp}}
\newcommand{\isSet}{\text{isSet}}
\newcommand{\isGrpd}{\text{isGrpd}}

\newcommand{\Contr}{\text{Contr}}
\newcommand{\Prop}{\text{Prop}}
\newcommand{\Set}{\text{Set}}
\newcommand{\Grpd}{\text{Grpd}}
%\newcommand{\Type}{\text{Type}}

% Functions.
\newcommand{\lam}[2]{\lambda #1.#2}
\newcommand{\apl}[2]{#1\ #2}

\newcommand{\id}{\text{id}}
\newcommand{\comp}{\circ}
\newcommand{\finv}[1]{#1^{-1}}

% Paths.
\newcommand{\eq}[2]{#1 = #2}
%\newcommand{\refl}[1]{\text{refl}_{#1}}
\newcommand{\refl}{\txt{refl}}

\newcommand{\inv}[1]{#1^{-1}}
\newcommand{\sq}{\mathbin{\vcenter{\hbox{\rule{.3ex}{.3ex}}}}}

% Equivalence.
\newcommand{\hequiv}[2]{#1 \simeq #2}

% Path application and transport.
\newcommand{\ap}[2]{\text{ap}_{#1}(#2)}
\newcommand{\apd}[2]{\text{apd}_{#1}(#2)}
\newcommand{\transport}{\text{transport}}

\newcommand{\qinv}{\text{qinv}}
\newcommand{\isequiv}{\text{isequiv}}

% Basic types.
\newcommand{\Empty}{\mathbf{0}}
\newcommand{\Unit}{\mathbf{1}}

\newcommand{\Bool}{\mathbf{2}}
\newcommand{\true}{\txt{tt}}
\newcommand{\false}{\txt{ff}}

\newcommand{\Nat}{\mathbb{N}}
\newcommand{\dprod}[2]{\prod #1.#2}
\newcommand{\dsum}[2]{\sum #1.#2}

% Products.
\newcommand{\prodt}[2]{#1 \times #2}
\newcommand{\pair}[2]{(#1, #2)}
\newcommand{\outl}{\text{pr}_1}
\newcommand{\outr}{\text{pr}_2}

% Sums
\newcommand{\sumt}[2]{#1 + #2}
\newcommand{\inl}{\txt{inl}}
\newcommand{\inr}{\txt{inr}}

% Logic.
\newcommand{\all}[2]{\forall #1.#2}
\newcommand{\ex}[2]{\exists #1.#2}
\newcommand{\conj}[2]{#1 \land #2}
\newcommand{\disj}[2]{#1 \lor #2}
\newcommand{\True}{\top}
\newcommand{\False}{\bot}
\newcommand{\impl}[2]{#1 \implies #2}

% Rest.
\newcommand{\happly}{\text{happly}}
\newcommand{\funext}{\text{funext}}

\newcommand{\idtoeqv}{\text{idtoeqv}}
\newcommand{\ua}{\text{ua}}

\newcommand{\code}{\text{code}}
\newcommand{\encode}{\text{encode}}
\newcommand{\decode}{\text{decode}}

\newcommand{\hS}{\mathbb{S}^1}
\newcommand{\base}{\text{base}}
\newcommand{\looop}{\text{loop}}

\newcommand{\I}{\mathbb{I}}
\newcommand{\IZ}{0_\mathbb{I}}
\newcommand{\II}{1_\mathbb{I}}
\newcommand{\seg}{\text{seg}}

\newcommand{\trf}[1]{||#1||}
\newcommand{\tri}[1]{|#1|}

\title{My HoTT notes}
\date{12 June 2020 - ongoing}

\begin{document}

\maketitle

\newcommand{\Searchable}{\txt{Searchable}}

\begin{center}
$\displaystyle \apl{\Searchable}{A} \defn \dprod{p : A \to \Bool}{\(\dsum{x : A}{\apl{p}{x} = \true}\) + \dprod{x : A}{\apl{p}{x} = \false}}$
\end{center}

In words: the type $A$ is searchable if for every boolean predicate $p$ we can find an element of $A$ that satisfies $p$ or else prove that no element of $A$ satisfies $p$.

\begin{theorem}
The type $\apl{\Searchable}{A}$ can be of any h-level.
\end{theorem}
\begin{proof}
$\apl{p}{x} = \true$ and $\apl{p}{x} = \false$ are propositions, the summands are disjoint, and dependent products preserve h-level (well, unless the domain is $\Empty$), so the h-level is determined by $A$.

If $A$ is contractible, then $\apl{\Searchable}{A}$ is contractible.

If $A$ is a proposition, then $\apl{\Searchable}{A}$ is a proposition.

If $A$ is of h-level $n$, then so is $\apl{\Searchable}{A}$.

Hope my mental techniques for h-levels work :)

\end{proof}

\newcommand{\MerelySearchable}{\txt{MerelySearchable}}
\begin{center}
$\displaystyle \apl{\MerelySearchable}{A} \defn \all{p : A \to \Bool}{\(\ex{x : A}{\apl{p}{x} = \true}\) \lor \all{x : A}{\apl{p}{x} = \false}}$
\end{center}

\begin{theorem}
$\apl{\isProp}{(\MerelySearchable{A})}$
\end{theorem}
\begin{proof}
We use the truncated logic.
\end{proof}

\begin{theorem}
Searchable types include: $\Empty, \Unit, \Bool$, all finite types and the interval.

If $A$ and $B$ are searchable, so are $\prodt{A}{B}$ and $A + B$.

If $A$ is searchable and $B : A \to \U$ is a family of searchable types, $\displaystyle \dsum{x : A}{\apl{B}{x}}$ is also searchable.
\end{theorem}
\begin{proof}
For finite types: we assume that ``finite'' means ``there is a list of all the elements''. So, use it to check every element, and poof -- done.

The interval is contractible, thus equivalent to $\Unit$, so it's searchable.

To search $A + B$ first search $A$, then search $B$.

To search $\prodt{A}{B}$ and $\dsum{x : A}{\apl{B}{x}}$ search for an $a$, and when looking for it, search for a $b$ that can be paired with it.
\end{proof}

\begin{theorem}
Searchability of $\Nat$ is taboo.

In general, preservation of searchability for $\prod, \txt{W}$ and $\txt{M}$ should be taboo.
\end{theorem}
\begin{proof}
$\Nat$ is the primordial searchability taboo -- how can we search infinitely many numbers in a finite time (besides checking every next number twice as fast as the previous one)?

\txt{W} can be used to define $\Nat$, so its searchability should be taboo.



\end{proof}

\end{document}